
\section{Input Streams}\label{sec:streams}
Public and private inputs are provided as separate resources. 
We have one input file per type and per visibility level (\texttt{public\_input} or \texttt{private\_input}).
These input files start with the same headers as described in \cref{sec:headers}: the version, the resource type (\texttt{public\_input} or \texttt{private\_input}), and one type.
Then, a sequence of numeric literals representing type elements is provided between \texttt{@begin} and \texttt{@end} tags.
These sequences act as a stream, and certain directives in the circuit % relation
consume a value from one of these streams.
If values in either stream are exhausted, this is a failure of evaluation validity.
If values remain in a stream after processing, then this is also an evaluation invalidity.
Here is an example for public and private inputs. 

\begin{lstlisting}[language=ir]
version 2.0.0;
public_input;
@type field 7;
@begin
  < 5 >;
@end
\end{lstlisting}
%
\begin{lstlisting}[language=ir]
version 2.0.0;
public_input;
@type field 19;
@begin
  < 2 >;
  < 15 >;
@end
\end{lstlisting}
%
\begin{lstlisting}[language=ir]
version 2.0.0;
private_input;
@type field 7;
@begin
  < 3 >;
  < 4 >;
@end
\end{lstlisting}
