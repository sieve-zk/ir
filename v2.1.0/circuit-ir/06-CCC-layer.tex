
\section{Circuit Configuration Communication (CCC)}

\subsection{Motivation}

To ease interoperation each backend should provide a configuration file declaring available types and plugins that the frontend may use.
This \textit{Circuit Configuration Communication}, or \textit{CCC} for short, is a static configuration file provided with the installation of a backend, and the frontend can configure from it while generating a statement.
The CCC starts with supported plugins, then types, conversions, and it may end with plugin constraints.
A compatible frontend will only use
\begin{itemize}
    \item the types supported by the targeted backend according to the CCC file,
    \item the standard gates (\Cref{subsec:standardgates}),
    \item the conversion gates supported by the targeted backend according to the CCC file, and
    \item the plugins supported by the targeted backend according to the CCC file.
\end{itemize}

All SIEVE IR compatible backends must provide a CCC, and all SIEVE IR compatible frontends must ingest a backend's CCC to generate a statement for the backend.
In the case that a backend cannot provide a feature (such as a type or a plugin) which the frontend would require, the frontend will be unable to generate a statement, and the pair of frontend and backend would be mutually incompatible.

\subsection{CCC content and syntax}

% 
A CCC file starts with the standard IR header (\Cref{sec:headers}), declaring its resource type as \texttt{configuration}.
It list the plugins, families of types, and conversion gates supported by the backend.

Each available plugin is specified by name using the same \texttt{@plugin plugin\_name;} syntax as the Circuit IR.
A plugins presence in the CCC indicates its availability, meaning that the frontend is allowed to use it, but it does not require the frontend to use it.
The frontend may use any subset of the backend's available plugins.
Similarly, a backend may indicate that no plugins are supported, by leaving the plugins list empty.

Here is an example plugin list indicating the availability of the \verb|mux_v0|, \verb|arith_ram_v0|, and \verb|arith_ram_v1| plugins.
Presumably, a frontend could then choose if to use the \verb|mux_v0| plugin, and if it needs RAM it could choose either version of the ram plugin.
The frontend could also use both versions of the RAM plugin, although that would be redundant and confusing.

\begin{lstlisting}[language=ir]
@plugin mux_v0;
@plugin ram_arith_v0;
@plugin ram_arith_v1;
\end{lstlisting}

\subsection{Type Families}\label{sec:families}

Types are grouped into ``type families'' in the CCC.
Each type family encompasses a set of types which are somehow related.
For example all fields which use Mersenne primes may be grouped into a ``Mersenne family''.

Each type family is specified first by what kind of type it is, then by predicates which constrain the set of possible types in the family.
A type family may be referenced later in the CCC by its index in the order in which they appear, same as how Circuit IR types are referenced.
There are four kinds of types, each having a different mathematical structure and accordingly different meanings for predicates.

\begin{itemize}
  \item A \texttt{field} type is defined by a single prime characteristic.
    Implicitly, the characteristic must be prime, and the predicates of a \texttt{field} family enforce further restrictions on it.
    A \texttt{field} family has the form \texttt{@type field(\textit{predicate} [, \textit{predicates...}]);}.
    An empty list of predicates (e.g. empty parenthesis) indicates that any prime is supported (including, for example, 7).
  \item An \texttt{ext\_field} type is defined by a base field, a polynomial order, and a numeric encoding of polynomial coefficients (see \Cref{sec:types}).
    There are three groups of predicates to an \texttt{ext\_field} family.
    The first group is a list of previously declared base \texttt{field} families, referenced by index.
    The base type of a member of this \texttt{ext\_field} family must be a member of one of the allowed base \texttt{field} families, but need not be a member of all (in fact, two base \texttt{field} families may be distinct of each other).
    The second group is predicates upon the order of the extension field.
    The order must be greater than one, and predicates may define additional restrictions.
    A third group defines predicates upon the numerically encoded polynomial coefficients.
    The polynomial must be irreducible, and predicates may define additional restrictions.
    An \texttt{ext\_field} family has the form \texttt{@type ext\_field (\textit{base\_idx} [, \textit{base\_idxs...}]) (\textit{predicate} [, \textit{predicates...}]) (\textit{predicate} [, \textit{predicates...}]);}.
    The family must define at least one \texttt{\textit{base\_idx}}, but both the order predicates and polynomial predicates may be empty.
  \item A \texttt{ring} type uses a ring over a $2^n$ modulus, providing the familiar bit representation for n-bit integers.
    The predicates of a \texttt{ring} family enforce restrictions on $n$.
    A \texttt{ring} family has the form \texttt{@type ring(\textit{predicate} [, \textit{predicates...}]);}.
    An empty list of predicates (e.g. empty parenthesis) indicates that any $n$ is supported.
  \item A \texttt{plugin} type is defined using plugins. Plugin type families use plugin constraints (see below) instead of predicates.
    A \texttt{plugin} family has the form \texttt{@type @plugin \textit{plugin\_name};}.
\end{itemize}

\subsection{Predicates}\label{sec:predicates}

Predicates define constraints upon numeric parameters of a type.
The following predicates exist.

\begin{itemize}
  \item \texttt{less\_than( numeric )} constructs a predicate requiring the parameter to be less than the specified value.
  \item \texttt{greater\_than( numeric )} constructs a predicate requiring the parameter to be greater than the specified value.
  \item \texttt{equals( numeric )} constructs a predicate requiring the parameter to equal the specified value.
  \item \texttt{is\_mersenne} is a predicate requiring that the parameter be a Mersenne prime.
  \item \texttt{is\_proth} is a predicate requiring that the parameter be a Proth prime.
  \item \texttt{is\_power\_of\_2} is a predicate requiring that the parameter be a power of two.
\end{itemize}

When a type uses multiple predicates, each predicate is ANDed with the others.
To produce an OR relation between predicates, simply create multiple type families.

\subsection{Conversion Declarations}\label{sec:ccc_conv_decl}

Conversion gate declarations specify which type families the backend supports conversions between.
Conversion gates are declared with \texttt{@convert(@out: family\_index\_out, @in: family\_index\_in);}, where the type families converted to and from are specified by their indices.
It is currently not possible to configure constraints on the number of input or output wires in a conversion.
% 

\subsection{Plugin Constraints}\label{sec:plugin_constraints}

Certain plugins may define additional constraints, such as constraints on
vector widths or RAM address and value types.
For each such plugin that the backend supports, the CCC should contain a
\emph{constraint block} delimited by \texttt{@plugin\_constraint(ram) ... @end}, containing
\texttt{@constraint(...)} lines specific to that plugin.
The arguments to each \texttt{@constraint} line consist of comma-separated identifiers and
integers, similar to the arguments of a function's \texttt{@plugin} binding
declaration.
Predicates may also be used within a \texttt{@constraint}, however their associativity and combinations with arbitrary identifiers or integers is plugin defined.
Each plugin that uses constraints will specify the structure and semantics of
\texttt{@constraint} lines allowed within the plugin's constraint block.

\subsection{Example}\label{sec:ccc_example}

\begin{lstlisting}[language=ir]
version 2.0.0;
configuration;

// The following plugins may be used
@plugin mux_v0;
@plugin ram_arith_v0;
@plugin ram_arith_v1;

// 0: The field of exactly 2
@type field(equals(2));

// 1: Any prime greater than 1 million
@type field(greater_than(1000000));

// 2: Any mersenne prime which is greater than 100 and less than 1000.
// To avoid confusion, for a Mersenne Prime p such that p = 2**m - 1,
// 100 < p < 1000, rather than m.
@type field(is_mersenne, greater_than(100), less_than(1000));

// 3: An extension field over GF(2**4)
@type ext_field (0) (equals(4)) ();

// 4: a 16, 32, or 64-bit unsigned integer
@type ring(is_power_of_2, less_than(65), greater_than(15));

// 5, 6: RAM Types for either the v0 or v1 RAM plugin
@type @plugin ram_arith_v0;
@type @plugin ram_arith_v1;

// Conversion from large fields (type 1) to Booleans (type 0)
@convert(@out: 0, @in: 1);

// Conversion from large primes (type 1) to Mersennes (type 2)
@convert(@out: 2, @in: 1);

// Conversions from the extension field (type 3) to Booleans (type 0)
// Note, that the IR only defines conversions involving extension
// fields to or from their base fields.
@convert(@out: 0, @in: 3);

// Bidirectional conversions of fields (type 1) and rings (type 4)
@convert(@out: 4, @in: 1);
@convert(@out: 1, @in: 4);

// Note that different versions of the same plugin are
// unlikely to be compatible
// INVALID: @convert(@out: 5, @in: 6);

// Note that although constraints are shown for the vectors_v1
// plugin, they are shown for demonstration purpose and are not
// standardized. Frontends are unlikely to recognize these
// constraints.

@plugin_constraints(vectors_v1)
  // Element type must match either of the first two @field lines.
  // The plugin spec defines the meaning of repeated element_type
  // constraints; in this case, we assume repeated lines are
  // allowed and that the constraint is satisfied if at least one
  // line is satisfied.
  @constraint(element_type, 0);
  @constraint(element_type, 1);

  // Vector width must be a power of two between 2 and 16.  The
  // meaning of these constraints is defined by the plugin spec.
  @constraint(vector_width, is_power_of_2, 2, 16);
@end
\end{lstlisting}

