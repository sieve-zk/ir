\newpage
\section{IR Overview and Abstract Syntax}\label{overview}
This section will describe the SIEVE IR at an overview level and should be sufficient to grant a workable understanding of the IR.
Section \ref{text} will provide authoritative textual syntax and IR semantics; Section \ref{binary} will provide binary syntax.
Mentioned previously, the IR has three (3) ``resources'' which may be thought of as files but could also be provided via other methods -- e.g. streams, shared memory, and so on.
These resources take the R(x; w) form: R is the relation, x is the instance, and w is the witness. \\

The IR's relation represents a circuit using sequence of directives to manage a sparsely populated lists of wire values.
A value is described by its index in the list, and most directives will insert a new value at an output index.
A wire's index is represented as an integer in the range of $0$ and $2^{64}-1$.
Although a directive may reference or insert a value at any index, using sequential indexes can improve performance.
The lists are considered sparse for two reasons. 
First, wires may be removed from the list to conserve memory.
Second, the minimal constraints on output index usage means that gaps may form in the wires list, although this is discouraged.\\

The instance and witness resources are streams of values.
Certain directives in the relation will consume a value from either stream.\\

\lstset{language=ir}
\lstset{frame=single}
\lstset{escapechar=\#}

\newcommand{\asTemplate}[1]{{\color{SyntaxGreen}{\textit{\{\texttt{#1}\}}}}}
\newcommand{\asRepeat}[0]{{\color{SyntaxGreen}{...}}}

This overview uses ``template'' texts, surrounded with boxes, and where ``template expressions'' may be repeated and replaced by concrete expressions described elsewhere.
In general, whitespace is optional and ignored, except for a single whitespace element delimiting non-obvious token sequences.
Template expressions are italicized with surrounding delimiters with dark green text, for example \asTemplate{expression}.
All template expressions will be described as appropriate literal values, or sub expressions.
Repetition of a template expression is indicated with an ellipsis (\asRepeat), also in green.
Tables following each template will describe the semantics of each template.\\

\subsection{Resource Header Overview}\label{header_overview}
The first component of the IR is a header block with evaluation parameters for the circuit.
The header is shared by all resources.
There are two parameters defined by the header: a version number for quick recognition of the IR, and the computation field.\\

\begin{lstlisting}
version #\asTemplate{major}#.#\asTemplate{minor}#.#\asTemplate{patch}#;

field
  characteristic #\asTemplate{p}#
  degree 1;
\end{lstlisting}

\subsubsection*{Header Semantics}

\noindent
\begin{tabularx}{\textwidth}{|p{1in}|p{1in}|X|}
  \hline
  \textbf{Template Expression} & \textbf{Name} & \textbf{Description} \\
  \hline
  \asTemplate{major}\newline
  \asTemplate{minor}\newline
  \asTemplate{patch}\newline & Version & The version of the IR is given in \href{https://semver.org/}{semantic version form}.
  This must match one of the versions given in the \nameref{history} section of this document. \newline
  \newline
  Syntactically, these parameters must be positive unbounded integers, in decimal form, with no leading zeros.\\
  \hline
  \asTemplate{p} & Field & The field governs the domain in which proofs may be expressed.
  It may be either $GF(p)$ or $GF(2)$ (a special case of the former).
  The syntax gives a characteristic ($p$), and a degree ($n$) which would describe a field $GF(p^n)$.
  However the degree is fixed to one (1).\newline
  \newline
  The $p$ variable may be an unbounded integer, and must be prime. \\
  \hline
\end{tabularx}

\subsection{Instance and Short Witness Overview}\label{ins_wit_overview}
Syntactically each of the instance and short witness is a sequence of field elements, acting as a stream.
Certain directives in the relation will consume a value from one of these streams.
If values in either stream are exhausted this is a failure of \textit{evaluation validity}.
If values remain in a stream after processing then this is also an \textit{evaluation invalidity}.\\

\begin{lstlisting}
#\asTemplate{header}#

#\asTemplate{type}# @begin
  < #\asTemplate{c}# > ; #\asRepeat#
@end
\end{lstlisting}

\subsubsection*{Instance and Witness Semantics}

\noindent
\begin{tabularx}{\textwidth}{|p{1in}|X|p{1.5in}|p{1.75in}|}
  \hline
  \textbf{Template Expression}
  & \textbf{Description}
  & \textbf{Syntactic\newline Constraints}
  & \textbf{Semantic\newline Constraints} \\
  \hline
  \asTemplate{header}
  & A header, as described in section \ref{header_overview}, is required here.
  & See section \ref{header_overview}.
  & The $\{p\}$ variable is captured in this section. \\
  \hline
  \asTemplate{type}
  & Indicates whether this resource is the instance or witness.
  & Must be the exact text \texttt{instance} or \texttt{short\_witness}. & \\
  \hline
  \asTemplate{c}\asRepeat
  & Each \asTemplate{c} is a literal constant specified as a gate input.
  & A numeric literal, which is syntactically positive and unbounded.
  & Must be in the range of $0$ up to $\{p\}$. \\
  \hline
\end{tabularx}

\subsection{Relation Overview: Structure, Gate Set, and Feature Toggles}\label{rel_structure_overview}
The relation starts by describing its own contents, first by which gates it may use (the \textit{gate set}) then by which uniformity features it may use (\textit{feature toggles}).
Following these is a list of function gate declarations, a list of the named function gates allowed in the relation, along with function gate definitions.
Lastly is the body of  the relation, the list of directives making up the circuit.\\

\begin{lstlisting}
#\asTemplate{header}#
relation 
gate_set: #\asTemplate{gateset}#;
features: #\asTemplate{features}#;
@begin
  #\asTemplate{functions}\asRepeat#
  #\asTemplate{directive}\asRepeat#
@end
\end{lstlisting}

Each template expression is described in this table.\\

\noindent
\begin{tabularx}{\textwidth}{|p{1.25in}|p{1in}|X|}
  \hline
  \textbf{Template \newline Expression} & \textbf{Name} & \textbf{Description} \\
  \hline
  \asTemplate{header}
  & Header
  & A header, as described in section \ref{header_overview}, is required here. \\
  \hline
  \asTemplate{gateset}
  & \textit{Gate set}
  & The \textit{gate set} describes the allowable gate directives of the circuit.
  The actual gate directives will be described in section \ref{function_gate_overview}.
  There are two \textit{canonical gate sets}, either \texttt{arithmetic} or \texttt{boolean}.
  If a \texttt{canonical gate set} is not desired, then a \textit{partial gate set} may described as the enumeration of gate names.\newline
  \newline
  A \textit{partial gate set} is defined by a comma separated subset of the gates from a \textit{canonical gate set}.
  For \texttt{arithmetic}, these would be \texttt{@add}, \texttt{@addc}, \texttt{@mul}, and \texttt{@mulc}; and for \texttt{boolean} gates these would be \texttt{@and}, \texttt{@not}, and \texttt{@xor}.
  If an \textit{partial gate set} enumeration were empty, it would be considered a semantic error.
  \\
  \hline
  \asTemplate{features}
  & \textit{Feature Toggles}
  & The feature toggles are a list of optional features which may be enabled or disabled by their presence.
  If a feature is present, it must appear in a comma separated set. The allowed features are \texttt{@function}, \texttt{@for}, and \texttt{@switch}\newline
  \newline The toggles for \texttt{@for} and \texttt{@switch} each describes whether for loops (subsection \ref{loop_overview}) or switch-case statements (subsection \ref{switch_overview}) are enabled.
  The toggle for \texttt{@function} indicates whether function gates (subsection \ref{function_gate_overview}) are allowed, except for anonymous functions within the body of a for loop or switch-case statement.
  If no features are enabled, then the keyword \texttt{simple} should take the feature list's place.\newline
  \newline
  Unless the \textit{feature toggle} is \texttt{simple}, wires between $2^{63}$ and $2^{64} - 1$ are reserved as ``ephemeral wires'', meaning that the frontend may not emit them.
  Instead, the backend may insert them as needed.\\
  \hline
  \asTemplate{functions}\asRepeat
  & Function Gate Declarations
  & This is a list of function gate declarations.
  Each declares a named function gate which may be used as a \asTemplate{directive}.
  Further description will be given in subsection \ref{function_invoke_overview}\\
  \hline
  \asTemplate{directive}\asRepeat & Directives & This is a list of directives, each directive will be described in further subsections \ref{rel_gates_overview}, \ref{function_gate_overview}, \ref{loop_overview}, and \ref{switch_overview}. \\
  \hline
\end{tabularx}

\subsection{Relation Overview: Simple Gate Directives}\label{rel_gates_overview}
Most of the relation is described through lists of directives.
``Gate directives'' perform calculations on their inputs and produce an output.
In general, gate directives will look like the assignment of the calculation's result to a wire.
There are three allowed forms for a gate directive.\\

\begin{itemize}
  \item \textbf{Binary} directives have two wire inputs, and one output. They are written as
  \begin{lstlisting}
  $#\asTemplate{x}# <- #\asTemplate{calculation}# ( $#\asTemplate{y}# , $#\asTemplate{z}# );
  \end{lstlisting}
  \item \textbf{Binary constant} directives have one wire input, one constant input, and one output. They are written as
  \begin{lstlisting}
  $#\asTemplate{x}# <- #\asTemplate{calculation}# ( $#\asTemplate{y}# , < #\asTemplate{c}# > );
  \end{lstlisting}
  \item \textbf{Unary} directives have one wire input, and one output. They are written as
  \begin{lstlisting}
  $#\asTemplate{x}# <- #\asTemplate{calculation}# ( $#\asTemplate{y}# );
  \end{lstlisting}
\end{itemize}

The allowed \asTemplate{calculation}s for each directive vary by the profile given in the header.
Each of the given function names can be considered a keyword in the IR language. \\

\begin{itemize}
  \item Under the \texttt{boolean} \textit{canonical gate set}, the following calculations are defined.
  \begin{itemize}
    \item \texttt{@and} is a binary directive which computes logical and.
    \item \texttt{@xor} is a binary directive which computes logical exclusive or.
    \item \texttt{@not} is a unary directive which computes logical negation.
  \end{itemize}
  \item Under the \texttt{arithmetic} \textit{canonical gate set}, the following calculations are defined.
  \begin{itemize}
    \item \texttt{@mul} is a binary directive which computes field multiplication.
    \item \texttt{@add} is a binary directive which computes field addition.
    \item \texttt{@mulc} is a binary constant directive which computes field multiplication.
    \item \texttt{@addc} is a binary constant directive which computes field addition.
  \end{itemize}
\end{itemize}

In both profiles, five special directives, called the \emph{common directives}, are allowed: Copy, Assignment, Assert Zero, Input (instance and short\_witness), and Deletion, as indicated below.
Regardless of \textit{gate set}, these directives are always allowed.
\begin{itemize}
  \item \textbf{Copy} will duplicate the value of one variable into another variable.
  \begin{lstlisting}
  $#\asTemplate{x}# <- $#\asTemplate{y}#;
  \end{lstlisting}
  \item \textbf{Assignment} will give one variable a constant value.
  \begin{lstlisting}
  $#\asTemplate{x}# <- < #\asTemplate{c}# >;
  \end{lstlisting}
  \item \textbf{Assert Zero} will check that a variable is equivalent to zero (0), otherwise it constitutes an \textit{evaluation invalidity}.
  \begin{lstlisting}
  @assert_zero( $#\asTemplate{y}# );
  \end{lstlisting}
  \item \textbf{Input} directives have a single output wire, and read the next value of the named stream.
    The \asTemplate{input-stream} may be either of \verb|@short_witness| or \verb|@instance|, indicating which of the input streams to read from.
    These may be considered keywords of the IR language.
  \begin{lstlisting}
  $#\asTemplate{x}# <- #\asTemplate{input-stream}#;
  \end{lstlisting}
  \item \textbf{Deletion} will remove a wire or a range of wires from the wires list.
  \begin{lstlisting}
  @delete( $#\asTemplate{y}# );
  @delete( $#\asTemplate{a}#, $#\asTemplate{b}# );
  \end{lstlisting}
\end{itemize}

The following table describes each of the template expressions used in this subsection.\\

\noindent
\begin{tabularx}{\textwidth}{|p{1in}|X|p{1.5in}|p{1.75in}|}
  \hline
  \textbf{Template Expression}
  & \textbf{Description}
  & \textbf{Syntactic\newline Constraints}
  & \textbf{Semantic\newline Constraints} \\
  \hline
  \asTemplate{x}
  & A single directive output wire specified as a 0-based index.
  & A numeric literal in the range of $[0; 2^{64})$.
  & No \asTemplate{x} wire may appear more than once, even if the \asTemplate{x} has been deleted previously.\newline
  \newline
  Unless the \textit{feature toggle} is \texttt{simple}, \asTemplate{x} must not exceed $2^{63}$\\
  \hline
  \asTemplate{y} \newline
  \asTemplate{z}
  & A single directive input wire specified as a 0-based index.
  & A numeric literal in the range of $[0; 2^{64})$.
  & A wire \asTemplate{y} or \asTemplate{z} must have been assigned a value before appearing as a \asTemplate{y} or a \asTemplate{z} and must not have been previously deleted.\newline
  \newline
  Unless the \textit{feature toggle} is \texttt{simple}, \asTemplate{y} or \asTemplate{z} must not exceed $2^{63}$\\
  \hline
  \asTemplate{c}
  & A literal constant specified as a gate input.
  & A numeric literal which is syntactically an unbounded positive integer.
  & \asTemplate{c} must be in the range of $0$ up to but not including \asTemplate{p}. \\
  \hline
  \asTemplate{a}, \newline
  \asTemplate{b}
  & A range of wires starting at \asTemplate{a} and concluding at \asTemplate{b}, both inclusive.
  & A numeric literal in the range of $[0; 2^{64})$.
  & Each wire between \asTemplate{a} and \asTemplate{b} must have previously been assigned a value and must not have been previously deleted.\newline
  \newline
  Unless the \textit{feature toggle} is \texttt{simple}, no wire in this range must exceed $2^{63}$.\\
  \hline
\end{tabularx}

\subsection{Relation Overview: Function Gates}\label{function_gate_overview}
The function gate feature adds a function declaration and two directives to the IR.

\begin{description}[labelindent=0.375in]
    \item[Declaration] Declares a named function gate for later use.
      These must be listed together at the beginning of the relation.
    \item[Invocation] Invokes a previously declared function gate.
    \item[Anonymous Invocation] Define a function gate at the point of its invocation.
\end{description}

Should a backend wish not to support function gates directly, they may be ``inlined'', replacing their invocations with a copy of their body.
A procedure for this is outlined in section \ref{inlining}.\\

\subsubsection*{Function Gate Declaration}\label{function_declare_overview}
Function gates are declared ahead of time and invoked later.
Their declarations indicate the number of wires entering and exiting the function-gate, as well as the number of instance and short witness values it will consume.\\

\begin{lstlisting}
@function(#\asTemplate{name}#, @out: #\asTemplate{out}#, @in: #\asTemplate{in}#,
    @instance: #\asTemplate{n\_instance}#, @short_witness: #\asTemplate{n\_s\_witness}#)
  #\asTemplate{directives}\asRepeat#
@end
\end{lstlisting}

\subsubsection*{Function Gate Declaration Semantics}

\noindent
\begin{tabularx}{\textwidth}{|p{1.25in}|X|p{1.5in}|p{1.375in}|}
  \hline
  \textbf{Template\newline Expression}
  & \textbf{Description}
  & \textbf{Syntactic\newline Constraints}
  & \textbf{Semantic\newline Constraints} \\
  \hline
  \asTemplate{name}
  & The name by which this function gate will be referenced.
  & This is a C-like identifier, but may also use \texttt{.} or \texttt{::} for grouping similar functions.
  & The name must not have previously been used in another function gate declaration. \\
  \hline
  \asTemplate{out}
  & The number of outputs wires from the subcircuit.
  & Integer literal between $0$ and $2^{64}-1$.
  & Gate directives may output on wires $0$ through $\asTemplate{out} - 1$ to assign the subcircuit's output wires.\\
  \hline
  \asTemplate{in}
  & The number of input wires from the subcircuit.
  & Integer literal between $0$ and $2^{64}-1$.
  & Gate directives may input from wires $\asTemplate{out}$ through $\asTemplate{in} + \asTemplate{out} -1$ to consume the subcircuit's input wires.\\
  \hline
  \asTemplate{n\_instance}
  & The number of instance values consumed the subcircuit.
  & Integer literal between $0$ and $2^{64}-1$.
  & \asTemplate{directives}\asRepeat\  must have enough \texttt{@instance} directives or function gate invocations so that the \texttt{@instance} directive count plus all the invocations' \asTemplate{n\_instance}s is \asTemplate{n\_instance}, exactly.\\
  \hline
  \asTemplate{n\_s\_witness}
  & The number of short witness values consumed by the subcircuit.
  & Integer literal between $0$ and $2^{64}-1$.
  & \asTemplate{directives}\asRepeat\ must have enough \texttt{@short\_witness} directives or function gate invocations so that the \texttt{@short\_witness} directive count plus all the invocations' \asTemplate{n\_s\_witness}s is \asTemplate{n\_s\_witness}, exactly.\\
  \hline
  \asTemplate{directives}\asRepeat
  & A list of directives (other than more function gate definitions) forming the body of the function gate.
  & These are the same directives from IR0 as allowed by the \textit{gate set}. Additionally, IR1 directives may be used, except for more function gate declarations.
  & Gate directives may assign to wires $\asTemplate{in} + \asTemplate{out}$ or higher to create wires local to the subcircuit.\\
  \hline
\end{tabularx}

\subsubsection*{Function Gate Invocation}\label{function_invoke_overview}
A function gate may be invoked as follows.
An invocation may occur within the top-level circuit, or within a subcircuit.
Note however, if a function-gate is invoked from within a function-gate, it must have been declared lexically ahead of the current one, specifically preventing direct or indirect recursion.\\

\begin{lstlisting}
#\asTemplate{out-list}# <- @call(#\asTemplate{name}#, #\asTemplate{in-list}#);
\end{lstlisting}

If a function gate has no inputs or no outputs, then the invocation may take one of the following forms.
An invocation could also have no inputs and no output wires, presumably if it consumed from \texttt{@short\_witness} or \texttt{@instance} for input, and used \texttt{@assert\_zero} in lieu of output wires.\\

\begin{lstlisting}
@call(#\asTemplate{name}#, #\asTemplate{in-list}#);

#\asTemplate{out-list}# <- @call(#\asTemplate{name}#);

@call(#\asTemplate{name}#);
\end{lstlisting}

\subsubsection*{Function Gate Invocation Semantics}
\noindent
\begin{tabularx}{\textwidth}{|p{1.25in}|X|p{1.5in}|p{1.375in}|}
  \hline
  \textbf{Template\newline Expression}
  & \textbf{Description}
  & \textbf{Syntactic\newline Constraints}
  & \textbf{Semantic\newline Constraints} \\
  \hline
  \asTemplate{name}
  & The name of the function to be invoked.
  & This is a C-like identifier, but may also use \texttt{.} or \texttt{::} for grouping similar functions.
  & The name must have previously been used as a function gate declaration.\\
  \hline
  \asTemplate{out-list}
  & A list of wires which will connect the output wires of the subcircuit.
  & Must be a wire list, as described below.
  & The number of wires in the list must be equal to the \asTemplate{out} parameter of the function gate declaration.
  Each wire in the list must be distinct.
  \newline\newline
  The wires in this list must not have previously been assigned.
  These wires will be assigned after the invocation.\\
  \hline
  \asTemplate{in-list}
  & A list of wires which will connect the output wires of the subcircuit.
  & Must be a wire list, as described below.
  & The number of wires in the list must be equal to the \asTemplate{in} parameter of the function gate declaration.
  Wires in this list may be duplicates, although this is not recommended.
  \newline\newline
  The wires in this list must have previously been assigned.\\
  \hline
\end{tabularx}

The wire list syntax used for \asTemplate{out-list} and \asTemplate{in-list} are lists of either single indexes or index ranges.
Multiple indexes and index ranges may be separated by commas.

\begin{description}[labelindent=0.375in]
    \item[Single] \texttt{\$\asTemplate{x}}, where \asTemplate{x} is an integer literal.
    The wire list will contain the wire \asTemplate{x}.
    \item[Range] \texttt{\$\asTemplate{x}...\$\asTemplate{y}}, where \asTemplate{x} and \asTemplate{y} are integer literals.
    The wire list will contain every wire between \asTemplate{x} and \asTemplate{y} (inclusive).
\end{description}

\subsubsection*{Function Gate Invocation Example}
This example demonstrates a simple function gate computing the sum of four inputs.

\begin{lstlisting}
version 1.0.0;
field characteristic 97 degree 1;
relation
gate_set: arithmetic;
features: @function;
@begin
  @function(add_4, @out: 1, @in: 4,
      @instance: 0, @short_witness: 0)
    $5 <- @add($1, $2);
    $6 <- @add($3, $4);
    $0 <- @add($5, $6);
  @end

  $0 <- @short_witness;
  $1 <- @instance;
  $2 <- @short_witness;
  $3 <- @instance;
  $4 <- @call(add_4, $0...$3);
  @assert_zero($4);
@end
\end{lstlisting}

\subsubsection*{Anonymous Invocations}
An anonymous invocation declares an unnamed function gate, and invokes it in the same directive.
It has the following form.\\

\begin{lstlisting}
#\asTemplate{out-list}# <- @anon_call(#\asTemplate{in-list}#,
    @instance: #\asTemplate{n\_instance}#, @short_witness: #\asTemplate{n\_s\_witness}#)
  #\asTemplate{directives}\asRepeat#
@end
\end{lstlisting}

Just like a named function invocation, an anonymous function may have no inputs or no outputs.

\begin{lstlisting}
@anon_call(#\asTemplate{in-list}#, @instance: #\asTemplate{n\_instance}#,
    @short_witness: #\asTemplate{n\_s\_witness}#)
  #\asTemplate{directives}\asRepeat#
@end

#\asTemplate{out-list}# <- @anon_call(@instance: #\asTemplate{n\_instance}#,
    @short_witness: #\asTemplate{n\_s\_witness}#)
  #\asTemplate{directives}\asRepeat#
@end

@anon_call(@instance: #\asTemplate{n\_instance}#, @short_witness: #\asTemplate{n\_s\_witness}#)
  #\asTemplate{directives}\asRepeat#
@end
\end{lstlisting}

The following template expressions are related to those of the Function Gate Declaration (\ref{function_declare_overview}) or Function Gate Invocation (\ref{function_invoke_overview}).

\noindent
\begin{tabularx}{\textwidth}{|p{1.5in}|X|}
  \hline
  \textbf{Template\newline Expression} & \textbf{Explanation} \\
  \hline
  \asTemplate{name} & The \asTemplate{name} is not used by an anonymous invocation. \\
  \hline
  \asTemplate{out}, \asTemplate{in} & These template expressions of a Declaration are replaced with the length of the \asTemplate{out-list} and \asTemplate{in-list} wire lists. \\
  \hline
  \asTemplate{n\_instance}, \asTemplate{n\_s\_witness} & These have the same usage in a Declaration and an Anonymous Invocation. \\
  \hline
  \asTemplate{directives}\asRepeat & This has the same usage in  a Declaration and an Anonymous Invocation. \\
  \hline
  \asTemplate{out-list} \asTemplate{in-list} & These have the same usage in an Invocation and an Anonymous Invocation. \\
  \hline
\end{tabularx}

\subsubsection*{Anonymous Invocation Example}
This example demonstrates an anonymous function computing the sum of four inputs.

\begin{lstlisting}
version 1.0.0;
field characteristic 97 degree 1;
relation
gate_set: arithmetic;
features: @function;
@begin
  $0 <- @short_witness;
  $1 <- @instance;
  $2 <- @short_witness;
  $3 <- @instance;
  $4 <- @anon_call($0...$3, @instance: 0, @short_witness: 0)
    $5 <- @add($1, $2);
    $6 <- @add($3, $4);
    $0 <- @add($5, $6);
  @end
  @assert_zero($4);
@end
\end{lstlisting}

\subsection{Relation Overview: For Loops}\label{loop_overview}
A for loop repeats the invocation of a function gate.
The invocation may be anonymous, but this is not required.
It has the following form.\\

Should a backend wish not to support for loops directly a procedure for ``unrolling'' a loop is described in section \ref{unrolling}.\\

\begin{lstlisting}
#\asTemplate{assign-list}# <- @for #\asTemplate{iterator}# @first #\asTemplate{first}# @last #\asTemplate{last}#
  #\asTemplate{invocation}#
@end
\end{lstlisting}

A for loop may omit the \asTemplate{assign-list} so long as its \asTemplate{invocation} has no output wires. In this case the \asTemplate{invocation} must also not have an \asTemplate{out-list}.\\

\begin{lstlisting}
@for #\asTemplate{iterator}# @first #\asTemplate{first}# @last #\asTemplate{last}#
  #\asTemplate{invocation}#
@end
\end{lstlisting}

\subsubsection*{For Loop Semantics}

\begin{tabularx}{\textwidth}{|p{1.125in}|X|p{1.375in}|p{1.75in}|}
  \hline
  \textbf{Template\newline Expression}
  & \textbf{Description}
  & \textbf{Syntactic\newline Constraints} & \textbf{Semantic\newline Constraints} \\
  \hline
  \asTemplate{assign-list}
  & A list of wires, in the scope containing the for loop, which are assigned by the for loop.
  & The same wire list form as used by Function Gate Invocation (\ref{function_invoke_overview}).
  & Before the loop, all wires in the list must be unassigned. After the loop all of them will have been assigned. \\
  \hline
  \asTemplate{iterator}, \asTemplate{first}, \asTemplate{last}
  & These control the number of iterations of the loop.\newline\newline
  \asTemplate{iterator} is a loop iterator (see below), while \asTemplate{first} and \asTemplate{last} are start and stop conditions.
  & \asTemplate{iterator} is an identifier (see below).
  \newline\newline
  \asTemplate{first} and \asTemplate{last} may be iterator expressions (see below).
  & \asTemplate{iterator} increments by one (1) on each iteration of the loop.\newline\newline
  On the first iteration, \asTemplate{iterator} carries the value \asTemplate{first}, and on the last it carries the value \asTemplate{last}. \\
   \hline
  \asTemplate{invocation} (and transitively \asTemplate{out-list} and \asTemplate{in-list})
  & This is a function gate invocation, anonymous or not.
  & The wire indexes in the invocation's \asTemplate{out-list} and \asTemplate{in-list} may be replaced with iterator expressions (see below).
  & The union of each iteration's \asTemplate{out-list} must form an equivalent set to the \asTemplate{assign-list}.
  Each iteration's \asTemplate{out-list} must be distinct from that of each other iteration. \\
  \hline
\end{tabularx}

\subsubsection*{Iterator Expressions}\label{overview-iterators}
Iterator expressions enable the input and output wires of a function gate invocation to differ on each iteration of the loop.
Each iterator expression is a simple arithmetic expressions using the loop \asTemplate{iterator}s of any lexically containing loop.
The expressions replace wire indexes for the \asTemplate{invocation}'s inputs and outputs.
An iterator expression can take one of the following forms, and represents the index of a wire, using integers between $0$ and $2^{64}-1$.
Loop \asTemplate{iterator}s have lexical scope.
Functionally, this means that on scope boundaries entering an anonymous function they are preserved, but entering a named function they are not.
For further clarity refer to section \ref{for_loops_text}. \\

\begin{itemize}
    \item A numeric constant.
    \item A ``loop iterator'', which refers to the value of a loops iteration counter.
    Syntactically this is a C-style identifier, but may also use \texttt{.} or \texttt{::} for grouping similar functions. See Appendix \ref{textblocks} for details.
    \item An expression of ``sub-expressions''. This takes the form \texttt{(\asTemplate{a} \asTemplate{op} \asTemplate{b})}. Where \asTemplate{a} and \asTemplate{b} are ``sub- expressions'', and \asTemplate{op} is one of the following operations.
    The operations have wraparound behavior from $2^{64}-1$ to $0$.
    \begin{itemize}
        \item \texttt{+}: integer addition.
        \item \texttt{-}: integer subtraction.
        \item \texttt{*}: integer multiplication.
    \end{itemize}
    \item A division constant expression. This takes the form \texttt{\asTemplate{a} / \asTemplate{b}} where \asTemplate{a} is a ``sub-expression'' and \asTemplate{b} is a numeric constant. The result is the 64-bit unsigned integer quotient.
\end{itemize}

\subsubsection*{Examples}
Here are a few example ``iterator-expressions'', as well an example of Fibonacci.
Note that for Fibonacci, the ``$n$'' is a public constant (the bounds of the loop).

\begin{itemize}
    \item \texttt{\$123} is the exact wire.
    \item \texttt{\$i} or \texttt{\$(i + 123)} indexes an ``array'' within a for-loop.
    \item \texttt{\$((i * 10) + j)}  indexes a ``square array'' within a pair of for-loops.
\end{itemize}

\begin{lstlisting}
$2...$10 <- @for i @first 2 @last 10
  $i <- @anon_call($(i - 1), $(i - 2),
      @instance: 0, @short_witness: 0)
    $0 <- @add($1, $2);
  @end
@end
\end{lstlisting}

To clarify lexical \asTemplate{iterator} scoping consider the following two example nested loop excerpts.
The first nests using anonymous functions for the outer loop body.
The second uses a named function, and because some of the inner loop's iterator expressions use the outer loop's iterator, an error is caused.\\

\begin{lstlisting}
 /* ... */ <- @for i @first 0 @last 10
   /* ... */ <- @anon_call(/* ... */)
     @for j @first 0 @last 10
       $((i * 10) + j) <- @anon_call(/* ... */)
         /* ... */
       @end
   @end
 @end
\end{lstlisting}

The equivalent using named functions is however erroneous.\\

\begin{lstlisting}
@function(foo, /* ... */)
  /* ... */ <- @for j @first 0 @last 10
    $((i * 10) + j) <- @anon_call(/* ... */) // #{\color{red} Error}#
      /* ... */
    @end
  @end
@end

/* ... */ <- @for i @first 0 @last 10
  /* ... */ <- @call(foo, /* ... */)
@end
\end{lstlisting}

\subsection{Relation Overview: Switch-Case Statements}\label{switch_overview}
The switch-case directive grants the IR the capability to evaluate branches, as is ubiquitous in conventional computing.
However, in order to protect the the short witness, performance characteristics cannot always be mirrored.
If the length or structure of one branch differs from other branches, then the verifier could use such information to learn about the witness.
Thus, every branch is processed, and only the result of the matching branch is chosen as the ``selected'' result. \\

Should a backend wish not to support a switch-case directly, a procedure for conversion to a selection circuit, or ``multiplexing'', is described in section \ref{multiplexing}.\\

\begin{lstlisting}
#\asTemplate{assign-list}# <- @switch ( $#\asTemplate{condition}# )
  @case < #\asTemplate{case}# > : #\asTemplate{invocation}# 
  @case < #\asTemplate{case}# > : #\asTemplate{invocation}# 
  @case < #\asTemplate{case}# > : #\asTemplate{invocation}# 
  #\asRepeat#
@end
\end{lstlisting}



\subsubsection*{Switch-Case Semantics}
The \asTemplate{condition} is the wire index of an assigned wire, whose value is matched against each \asTemplate{case}.
Where the \asTemplate{condition}'s value matches a \asTemplate{case}, the corresponding \asTemplate{invocation}'s assignments are kept as the \asTemplate{assign-list}'s assignments.\\

If any two \asTemplate{case}s have the same value, then the relation is semantically invalid.
If the \asTemplate{condition} does not match any \asTemplate{case}, the witnessed statement is invalid.\\

\begin{tabularx}{\textwidth}{|p{1.125in}|X|p{1.375in}|p{1.75in}|}
  \hline
  \textbf{Template\newline Expression}
  & \textbf{Description}
  & \textbf{Syntactic\newline Constraints} & \textbf{Semantic\newline Constraints} \\
  \hline
  \asTemplate{assign-list}
  & A list of wires, in the scope containing the switch-case, which are assigned by the switch-case.
  & The same wire list form as used by Function Gate Invocation (\ref{function_invoke_overview}).
  & Before the switch-case, all wires in the list must be unassigned. Afterwards all of them will have been assigned. \\
  \hline
  \asTemplate{condition}, \asTemplate{case}
  & These select the ``selected'' case of the switch-case statement.
  & The \asTemplate{condition} is a wire-number, prefixed by \texttt{\$}, and in the range of $0$ through $2^{63}-1$.\newline
  \newline
  Each \asTemplate{case} is a field element, delimited with \texttt{<} and \texttt{>}.
  & The \asTemplate{condition} must have been previously assigned, and each \asTemplate{case} must be in the range of $0$ through $p-1$.
  No \asTemplate{case} may be duplicated within a same switch-case.\newline
  \newline
  If the \asTemplate{condition}'s value is not matched by a case, this is an \textit{evaluation invalidity}.
  \\
   \hline
  \asTemplate{invocation} (and transitively \asTemplate{out-list}, \asTemplate{in-list}, and \asTemplate{directives})
  & Each case is backed by a function gate invocation, either named or anonymous.
  & The \asTemplate{out-list} and \texttt{<-} are omitted from each invocation, otherwise, these are syntactically identical to a named or anonymous function-gate invocation.
  & The \asTemplate{assign-list} must have the same length as the \asTemplate{out-list} would have.
  The ``selected'' case's \asTemplate{out-list} is replaced with the \asTemplate{assign-list}, while each other case's \asTemplate{out-list} is replaced with wires which can be considered discarded.\newline
  \newline
  A \asTemplate{directive} which has side-effects must have special meaning described in subsection \ref{switch_side_effects_overview}.
  \\
  \hline
\end{tabularx}

\subsubsection*{Side-Effect Directives in a Switch-Case Statement}\label{switch_side_effects_overview}
As certain directives have effects other than assignment of wires, their semantics within a switch-case must be modified.
Specifically \texttt{@assert\_zero} and the input directives, \texttt{@instance} and \texttt{@short\_witness}, have altered semantics.\\

The \texttt{@assert\_zero} directive causes an \textit{evaluation invalidity} if its given wire does not carry the value zero (0).
This effect is undesirable in non-selected cases, thus the effect is disabled in the body of each case, except the selected case. \\

The input directives (\texttt{@instance} and \texttt{@short\_witness}) cause streams to be advanced on each occurrence, and each branch might use a different number of inputs.
If the IR were to match exactly the expectation of conventional branching, then differing stream consumption counts could reveal the switch condition, breaking zero-knowledge.
However, if the IR were to trace each branch and consume every input on each branch, this could be wasteful and preclude certain interesting optimizations.
Instead, the totality of the switch-case must consume the maximum number of inputs regardless of which case.
This could be implemented by duplicating the stream before each case (demonstrated by the reference semantics in subsection \ref{switch_case_text}), rewinding between cases, or replacing input directives with function-gate inputs (outlined in subsection \ref{multiplexing}).
Note that due to ``maximum consumption'', regardless of which case is selected, the overall stream length does not vary with case selections.\\

Using the ``maximum consumption'' may seem to produce nonsensical values in non-selected cases.
This is allowable, because the nonsensical values are ignored as both function gate outputs and as \texttt{@assert\_zero} inputs.\\

It is worth mentioning that that there may be up to $p-1$ unique cases within a switch.
In some cases it may be undesirable, impractical, or impossible to enumerate each case of the switch-case.
This leaves open the possibility for a witnessed-statement to have a switch-case statement with out a selected branch.
This possibility is defined as a witnessed-statement invalidity, meaning that backends must be careful to avoid malicious behavior.\\

\subsubsection*{Example}
This function implements a vectorized/simd style private arithmetic operation, with trivial encoding: $0 \rightarrow mul$, $1 \rightarrow add$. Here the logic is fully unrolled for practicality due to the brevity of the implicit loop. Input vectors are stored in wires \$1...\$4 and \$5..\$8.\\

\begin{lstlisting}
 $9...$12 <- @switch ($0)
    @case <0>:
        @anon_call($1...$8,@instance:0,@short_witness:0)
        $0 <- @mul($4, $8)
        $1 <- @mul($5, $9)
        $2 <- @mul($6, $10)
        $3 <- @mul($7, $11)
        @end
    @case <1>:
        @anon_call($1...$8,@instance:0,@short_witness:0)
        $0 <- @add($4, $8)
        $1 <- @add($5, $9)
        $2 <- @add($6, $10)
        $3 <- @add($7, $11)
        @end
 @end
\end{lstlisting}

\subsection{Supporting additional Syntax}

In the following sections \ref{text} and \ref{binary}, we concretely specify the two syntax serializations so that proposed features or tradeoffs can be more easily expressed in a rigorous manner.
