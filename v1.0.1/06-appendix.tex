
\newpage
\begin{appendices}
\section{Textual Syntax Building Blocks}\label{textblocks}

This Appendix describes the building blocks used by section \ref{text} of this specification.
It will briefly overview the number formats and identifiers used by the SIEVE IR.
Throughout this section both tokens (terminal symbols) and nonterminal symbols are described, thus indications of the intended interpretation are given.\\

The IR defines binary, octal, decimal, and hexadecimal literals as interchangeably, with conventional prefixes \texttt{0b}, \texttt{0o}, and \texttt{0x} for binary, octal, and hexadecimal literals.
Decimal literals have no prefixes, but must not have leading zeros.
All numeric literals represent unbounded positive integers.
A complete attribute grammar is provided later in this appendix.\\

\begin{alltt}\ttSyn
  \ttNontermN{numeric-literal}{0} ::= \ttNontermN{binary-literal}{1} | \ttNontermN{octal-literal}{1}
                      | \ttNontermN{decimal-literal}{1} | \ttNontermN{hex-literal}{1}\ttSem
    \ttNontermN{numeric-literal}{0}.value := match \{
      case \ttNontermN{binary-literal}{1}: \ttNontermN{binary-literal}{1}.value;
      ...
      case \ttNontermN{hex-literal}{1}: \ttNontermN{hex-literal}{1}.value;
    \}
\end{alltt}

The IR uses numeric identifiers for wires, or \ttNonterm{wire-number}.
These carry an index into the  \semGlobal{wirset} described by section \ref{wireset_text}.
Although \ttNonterm{wire-number}s are written here in BNF, they are to be interpreted as tokens, not allowing space between elements.\\

\begin{alltt}\ttSyn
  \ttNontermN{wire-number}{0} ::= `\$' \ttNontermN{numeric-literal}{1}\ttSem
    \ttNontermN{wire-number}{0}.index := \ttNontermN{numeric-literal}{1}.value;
\end{alltt}

The IR defines the following syntax for a Field Element Literal.
The Field Element in general is an element of the Galois Field defined for the statement.
For $GF(p^1)$, this is a numeric literal $\bmod\ p$ encased with angle braces.
The \synNonterm{field-literal} should be interpreted as allowing spaces between elements. \\

\begin{alltt}\ttSyn
  \ttNontermN{field-literal}{0} ::= `<' \ttNontermN{numeric-literal}{1} `>'\ttSem
    fail_if(\ttNontermN{numeric-literal}{1}.value >= \ttGlobal{p});
    \ttNontermN{field-literal}{0}.value := \ttNontermN{numeric-literal}{1}.value;
\end{alltt}

For referencing previously defined function gates and loop iterators, the \synNonterm{label} is used as a lookup key.
Its form is intended to be similar to a C-style identifier, but allows ``grouping'' similar names with separators `\texttt{.}' and `\texttt{::}'.
It should be interpreted as a token, disallowing spaces.\\

\begin{alltt}\ttSyn
  \ttNonterm{label} ::= [`a'-`z' | `A'-`Z' | `_']  [`a'-`z' | `A'-`Z' | `0'-`9' | `_']*
           (
              [`.' | `::']
              [`a'-`z' | `A'-`Z' | `_']  [`a'-`z' | `A'-`Z' | `0'-`9' | `_']*
           )*
\end{alltt}

The full format for numeric literals is given here.
They take the form of positive, unbounded integers in one of binary, octal, decimal, or hexadecimal format.
To indicate binary, octal, and hexadecimal, a \texttt{0b}, \texttt{0o}, or \texttt{0x} prefix is used.
Decimal numbers have no prefix, but may not have leading zeros, to avoid confusion with C-Style octal numerals (leading zero, with no ``o'' distinguisher).
Although numeric literals are written here in BNF, they are to be interpreted as tokens, not allowing space between elements.\\

\begin{alltt}\ttSyn
  \ttNontermN{binary-literal}{0} ::= [ `0b' | `0B' ] [ \ttNontermN{binary-digit}{1..m} ]+\ttSem
    \ttNontermN{binary-literal}{0}.value := 0;
    for(i from 1 to m) \{
      \ttNontermN{binary-literal}{0}.value *= 2;
      \ttNontermN{binary-literal}{0}.value += \ttNontermN{binary-digit}{i}.value;
    \}\ttSyn
  \ttNontermN{binary-digit}{0} ::= `0' | `1'\ttSem
    \ttNontermN{binary-digit}{0}.value := match \{
      case `0': 0;
      case `1': 1;
    \}\ttSyn

  \ttNontermN{octal-literal}{0} ::= `0o' [ \ttNontermN{octal-digit}{1..m} ]+\ttSem
    \ttNontermN{octal-literal}{0}.value := 0;
    for(i from 1 to m) \{
      \ttNontermN{octal-literal}{0}.value *= 8;
      \ttNontermN{octal-literal}{0}.value += \ttNontermN{octal-digit}{i}.value;
    \}\ttSyn
  \ttNontermN{octal-digit}{0} ::= `0' | `1' | `2' | `3' | `4' | `5' | `6' | `7'\ttSem
    \ttNontermN{octal-digit}{0}.value := match \{
      case `0': 0;
      ...
      case `7': 7;
    \}\ttSyn


  \ttNontermN{decimal-literal}{0} ::= `0'\ttSem
    \ttNontermN{decimal-literal}{0}.value := 0;\ttSyn
  \ttNontermN{decimal-literal}{0} ::= \ttNontermN{no-zero-decimal-digit}{1} [ \ttNontermN{decimal-digit}{1..m} ]*\ttSem
    \ttNontermN{decimal-literal}{0}.value := \ttNontermN{no-zero-decimal-digit}{1};
    for(i from 1 to m) \{
      \ttNontermN{decimal-literal}{0}.value *= 10;
      \ttNontermN{decimal-literal}{0}.value += \ttNontermN{decimal-digit}{i}.value;
    \}\ttSyn
  \ttNontermN{no-zero-decimal-digit}{0} ::= `1' | `2' | `3' | `4' | `5' | `6' |
                             `7' | `8' | `9'\ttSem
    \ttNontermN{no-zero-decimal-digit}{0}.value := match \{
      case `1': 1;
      ...
      case `9': 9;
    \}\ttSyn
  \ttNontermN{decimal-digit}{0} ::= `0' | \ttNontermN{no-zero-decimal-digit}{1}\ttSem
    \ttNontermN{decimal-digit}{0} := match \{
      case `0': 0;
      case \ttNontermN{no-zero-decimal-digit}{1}: \ttNontermN{no-zero-decimal-digit}{1}.value;
    \}\ttSyn

  \ttNontermN{hex-literal}{0} ::= [ `0x' | `0X' ] [ \ttNontermN{hex-digit}{1..m} ]+\ttSem
    \ttNontermN{hex-literal}{0}.value := 0;
    for(i from 1 to m) \{
      \ttNontermN{hex-literal}{0}.value *= 16;
      \ttNontermN{hex-literal}{0}.value += \ttNontermN{hex-digit}{i}.value;
    \}\ttSyn
  \ttNontermN{hex-digit}{0} ::= `0' | `1' | `2' | `3' | `4' | `5' | `6' | `7'
                | `8' | `9' | `a' | `b' | `c' | `d' | `e' | `f'
                | `A' | `B' | `C' | `D' | `E' | `F'\ttSem
    \ttNontermN{hex-digit}{0}.value := match \{
      case `0': 0;
      ...
      case `9': 9;
      case `a': 10;
      ...
      case `f': 15;
      case `A': 10;
      ...
      case `F': 15;
    \}\ttSyn
\end{alltt}

The IR text format will also allow ``comments'' of the following form, which will be treated as whitespace during parsing.

\begin{alltt}\ttSyn
  \ttNonterm{comment} ::= `//' [ `\{any character except {\textbackslash}n\}' ]* `{\textbackslash}n'
  \ttNonterm{comment} ::= `/*' [ `\{any character except the sequence `*/'\}' ]* `*/'
\end{alltt}

\end{appendices}